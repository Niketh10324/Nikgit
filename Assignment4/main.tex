\let\negmedspace\undefined
\let\negthickspace\undefined
\documentclass[journal]{IEEEtran}
\usepackage[a5paper, margin=10mm, onecolumn]{geometry}
%\usepackage{lmodern} % Ensure lmodern is loaded for pdflatex
\usepackage{tfrupee} % Include tfrupee package

\setlength{\headheight}{1cm} % Set the height of the header box
\setlength{\headsep}{0mm}     % Set the distance between the header box and the top of the text

\usepackage{gvv-book}
\usepackage{gvv}
\usepackage{cite}
\usepackage{amsmath,amssymb,amsfonts,amsthm}
\usepackage{algorithmic}
\usepackage{graphicx}
\usepackage{textcomp}
\usepackage{xcolor}
\usepackage{txfonts}
\usepackage{listings}
\usepackage{enumitem}
\usepackage{mathtools}
\usepackage{gensymb}
\usepackage{comment}
\usepackage[breaklinks=true]{hyperref}
\usepackage{tkz-euclide} 
\usepackage{listings}
% \usepackage{gvv}                                        
\def\inputGnumericTable{}                                 
\usepackage[latin1]{inputenc}                                
\usepackage{color}                                            
\usepackage{array}                                            
\usepackage{longtable}                                       
\usepackage{calc}                                             
\usepackage{multirow}                                         
\usepackage{hhline}                                           
\usepackage{ifthen}                                           
\usepackage{lscape}


\renewcommand{\thefigure}{\theenumi}
\renewcommand{\thetable}{\theenumi}
\setlength{\intextsep}{10pt} % Space between text and floats


\numberwithin{equation}{enumi}
\numberwithin{figure}{enumi}
\renewcommand{\thetable}{\theenumi}


% Marks the beginning of the document
\begin{document}
\bibliographystyle{IEEEtran}
\vspace{3cm}

\title{jee-main-maths-06-04-2023-shift-2}
\author{EE24BTECH11047 - Niketh Prakash Achanta}
% \maketitle
% \newpage
% \bigskip
{\let\newpage\relax\maketitle}
\renewcommand{\thefigure}{\theenumi}
\renewcommand{\thetable}{\theenumi}
\begin{enumerate}[start=16]
    \item The sum of all values of $\alpha$, for which the points whose position vectors are $\hat{i}-2\hat{j}+3\hat{k}, 2\hat{i}-3\hat{j}+4\hat{k}, \brak{\alpha+1}\hat{i}+2\hat{k}$ and $9\hat{i}+\brak{\alpha-8}\hat{j}+6\hat{k}$ are coplanar, is equal to \\ 
    \begin{multicols}{4}
    \begin{enumerate}
        \item -2
        \item 2
        \item 6
        \item 4
    \end{enumerate}
    \end{multicols}
    \item Let the line $\vec{L}$ pass through the point \brak{0, 1, 2}, intersect the line $\frac{x-1}{2}=\frac{y-2}{3}=\frac{z-3}{4}$ and be parallel to the plane $2x+y-3z=4$. Then the distance of $\vec{P}\brak{1,-9,2}$ from the line $\vec{L}$ is \\
    \begin{multicols}{4}
    \begin{enumerate}
        \item $9$
        \item $\sqrt{54}$
        \item $\sqrt{69}$
        \item $\sqrt{74}$
    \end{enumerate} 
    \end{multicols}
    \item All the letters of the word PUBLIC are written in all possible orders and these words are written as in a dictionary with serial numbers. Then the serial number of the word PUBLIC is :   \\
    \begin{multicols}{4}
    \begin{enumerate}
        \item 580
        \item 578
        \item 576
        \item 582
    \end{enumerate} 
    \end{multicols}
    \item Let the vectors $\vec{a,b,c}$ represent three coterminous edges of a parallelepiped of volume V. Then the volume of the parallelepiped, whose coterminous edges are represented by $\vec{a,b+c}$ and $\vec{a+2b+3c}$ is equal to: \\
    \begin{multicols}{4} 
    \begin{enumerate}
        \item 2V
        \item 6V
        \item 3V
        \item V
    \end{enumerate} 
    \end{multicols}
    \item Among the statements : \\ 
    $\brak{S1}$ : $2023^{2022}-1999^{2022}$ is divisible by $8$ \\
    $\brak{S2}$ : $13\brak{13}^{n}-11n-13$ is divisible by 144 for infinitely many n $\in \mathbf{N}$
    \begin{multicols}{2}
    \begin{enumerate}
        \item only $\brak{S2}$ is correct
        \item only $\brak{S1}$ is correct
        \item both $\brak{S1}$ and $\brak{S2}$ are incorrect
        \item both $\brak{S1}$ and $\brak{S2}$ are correct
    \end{enumerate} 
    \end{multicols}
    \item The value of $\tan{9}^{\degree}-\tan{27}^{\degree}-\tan{63}^{\degree}+\tan{81}^{\degree}$ is: \\
    \newpage
    \item If $\brak{20}^{19}+2\brak{21}\brak{20}^{18}+3\brak{21}^{2}\brak{20}^{17}+\cdots+20\brak{21}^{19}=k\brak{20}^{19}$, then k is equal to \\

    \item Let the eccentricity of an ellipse $\frac{x^2}{a^2}+\frac{y^2}{b^2}=1$ is reciprocal to that of the hyperbola $2x^2-2y^2=1$. If the ellipse intersects the hyperbola at right angles, then square of length of the latus-rectum of the ellipse is: \\

    \item For $\alpha,\beta,z\in\mathbf{C}$ and $\lambda>1$, if $\sqrt{\lambda-1}$ is the radius of the circle $\abs{z-\alpha}^2+\abs{z-\beta}^2=2\lambda$, then $\abs{\alpha-\beta}$ is equal to\\

    \item Let a curve $y = f\brak{x}$,  $x\in\brak{0,\infty}$ pass through the points $\vec{P}\brak{1,\frac{3}{2}}$ and $\vec{Q}\brak{a,\frac{1}{2}}$. If the tangent at any point $\vec{R}\brak{b,f\brak{b}}$ to the given curve cuts the y-axis at the points $\vec{S}\brak{0,c}$ such that $bc=3$, then $\brak{PQ}^2$ is equal to\\

    \item If the lines $\frac{x-1}{2} = \frac{2 - y}{-3} = \frac{z - 3}{\alpha}$ and $\frac{x - 4}{5} = \frac{y - 1}{2} = \frac{z}{\beta}$ intersect, then the magnitude of the minimum value of $8 \alpha \beta$ is:\\

    \item Let $f\brak{x}=\frac{x}{\brak{1+x^{n}}^{1/n}}$, $x\in\mathbf{R}-\cbrak{-1},n\in N,n>2.$ If $f^n\brak{x}=n\brak{fofof\cdots \text{upto n times}}\brak{x}$, then\\
    $\lim_{n\to\infty}\int_{0}^{1} x^{n-2}\brak{f^n\brak{x}} \,dx$ is equal to\\

    \item If the mean and variance of the frequency distribution. \\
    \begin{tabular}{|c|c|c|c|c|c|c|c|c|} 
        \hline
            $x_i$ & 2 & 4 & 6 & 8 & 10 & 12 & 14 & 16 \\ 
        \hline
            $f_i$ & 4 & 4 & $\alpha$ & 15 & 8 & $\beta$ & 4 & 5 \\ 
        \hline
    \end{tabular}\\
    are 9 and 15.08 respectively, then the  value of $\alpha^2+\beta^2-\alpha\beta$ is\\

    \item The number of points, where the curve $y=x^{5}-20x^{3}+50x+2$ crosses the x-axis is:\\

    \item The number of 4-letter words, with or without meaning, each consisting of 2 vowels and 2 consonants, which can be formed from the letters of the word UNIVERSE without repetition is: \\


\end{enumerate}
\end{document}