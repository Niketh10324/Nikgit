%iffalse
\let\negmedspace\undefined
\let\negthickspace\undefined
\documentclass[journal,12pt,onecolumn]{IEEEtran}
\usepackage{cite}
\usepackage{amsmath,amssymb,amsfonts,amsthm}
\usepackage{algorithmic}
\usepackage{graphicx}
\usepackage{textcomp}
\usepackage{xcolor}
\usepackage{txfonts}
\usepackage{listings}
\usepackage{enumitem}
\usepackage{mathtools}
\usepackage{gensymb}
\usepackage{comment}
\usepackage[breaklinks=true]{hyperref}
\usepackage{tkz-euclide}
\usepackage{listings}
\usepackage{gvv}                                        
%\def\inputGnumericTable{}                                
\usepackage[latin1]{inputenc}                                
\usepackage{color}                                            
\usepackage{array}                                            
\usepackage{longtable}                                      
\usepackage{calc}                                            
\usepackage{multirow}                                        
\usepackage{hhline}                                          
\usepackage{ifthen}                                          
\usepackage{lscape}
\usepackage{tabularx}
\usepackage{array}
\usepackage{float}


\newtheorem{theorem}{Theorem}[section]
\newtheorem{problem}{Problem}
\newtheorem{proposition}{Proposition}[section]
\newtheorem{lemma}{Lemma}[section]
\newtheorem{corollary}[theorem]{Corollary}
\newtheorem{example}{Example}[section]
\newtheorem{definition}[problem]{Definition}
\newcommand{\BEQA}{\begin{eqnarray}}
\newcommand{\EEQA}{\end{eqnarray}}
\newcommand{\define}{\stackrel{\triangle}{=}}
\theoremstyle{remark}
\newtheorem{rem}{Remark}

% Marks the beginning of the document
\begin{document}
\bibliographystyle{IEEEtran}
\vspace{3cm}

\title{11. Limits, Continuity and Differentiability}
\author{EE24BTECH11047 - Niketh Prakash Achanta}
\maketitle
\bigskip
\section{C: MCQs with one correct answer}

\renewcommand{\thefigure}{\theenumi}
\renewcommand{\thetable}{\theenumi}
\begin{enumerate}
\item %17

	$\lim_{x\to0}$  $\frac{\sin\brak{\pi \cos^2 x }}{x^2}$ \: equals \hfill{\brak{2001S}}
    \begin{enumerate}
     \item $-\pi$
     \item $\pi$
     \item $\pi/2$
     \item $1$\\
     \end{enumerate}

\item %18
	The left-hand derivative of $f\brak{x}=\sbrak{x}\sin\brak{\pi x}$ at\ $x=k$, k an integer, is \hfill{\brak{2001S}}
    \begin{enumerate}
	    \item $\brak{-1}^k \brak{-1} \pi$
	    \item $\brak{-1}^{k-1}\brak{k-1} \pi$
            \item $-1^k k\pi$
     	    \item $-1^{k-1} k\pi$\\
    \end{enumerate}

\item %19

	Let $f:\mathbf{R}\rightarrow\mathbf{R}$ be a function defined by $f\brak{x}=$max$\cbrak{x,x^3}$.\ The set of all points where $f\brak{x}$ is NOT differentiable is \hfill{\brak{2001S}}
    \begin{enumerate}
     \item $\cbrak{-1,1}$
     \item $\cbrak{-1,0}$
     \item $\cbrak{0,1}$
     \item $\cbrak{-1,0,1}$\\
    \end{enumerate}


\item %20

	Which of the following functions is differentiable at $x=0$ \hfill{\brak{2001S}}
    \begin{enumerate}
	    \item $\cos\brak{\abs{x}}+\abs{x}$
	    \item $\cos\brak{\abs{x}}-\abs{x}$
	    \item $\sin\brak{\abs{x}}+\abs{x}$
	    \item $\sin\brak{\abs{x}}-\abs{x}$\\
    \end{enumerate}


\item %21

    The domain of the derivative of the function
		f$\brak{x}=$
 $\begin{cases}
	\tan^{-1}x, \text{if} \: \abs{x}\leq0 \\
    	\frac{1}{2}\brak{\abs{x}-1}, \text{if} \: x > 1 
\end{cases} $ \  \hfill{\brak{2002S}}
    \begin{enumerate}
	    \item $\mathbf{R}-\cbrak{0}$
	    \item $\mathbf{R}-\cbrak{1}$
	    \item $\mathbf{R}-\cbrak{-1}$
	    \item $\mathbf{R}-\cbrak{-1,1}$\\
    \end{enumerate}


\item %22

	The integer n for which $\lim_{x \to 0}$ $\frac{\brak{\cos x-1} \brak{\cos x-e^{x}}}{x^{n}}$ is a finite non-zero number is \hfill{\brak{2002S}}
    \begin{enumerate}
     \item $1$
     \item $2$
     \item $3$
     \item $4$\\
    \end{enumerate}

\item %23

	Let $f:\mathbf{R} \rightarrow \mathbf{R}$ be such that $f\brak{1}=3$ and $f'\brak{1}=6$. Then $\lim_{x \to 0}$ $ \brak{ \frac{f\brak{1+x}}{f\brak{1}}}^{1/x}$\ equals \hfill{\brak{2002S}}
    \begin{enumerate}
     \item $1$
     \item $e^{1/2}$
     \item $e^2$
     \item $e^3$\\
    \end{enumerate}


\item %24

	If $\lim_{x \to 0} \frac{\brak{\brak{a-n}nx-\tan x}\sin nx}{x^2} =0,$ where n is nonzero real number, then a is equal to\hfill{\brak{2003S}}
    \begin{enumerate}
     \item $0$
     \item $\frac{n+1}{n}$
     \item $n$
     \item $n+\frac{1}{n}$\\
    \end{enumerate}


\item %25

	$\lim_{h \to 0} \frac{f\brak{2h+2+h^2}-f\brak{2}}{f\brak{h-h^2+1}-f\brak{1}}$ given that $f'\brak{2}=6$ and $f'\brak{1}=4$ \hfill{\brak{2003S}}
    \begin{enumerate}
     \item Does not exist
     \item is equal to $-3/2$
     \item is equal to $3/2$
     \item is equal to $3$\\
    \end{enumerate}


\item %26

	If f\brak{x} is differentiable and strictly increasing function, then the value of $\lim_{x \to 0} \frac{f\brak{x^2}-f\brak{x}}{f\brak{x}-f\brak{0}}$ is \hfill{\brak{2004S}}
    \begin{enumerate}
	    \item $\brak{-1}^k \brak{k-1} \pi$
	    \item $\brak{-1}^{k-1} \brak{k-1} \pi$
	    \item $\brak{-1}^k k\pi$
            \item $\brak{-1}^{k-1} k\pi$\\
    \end{enumerate}


\item %27

	The function given by $y=\abs{\abs{x}-1}$ is differentiable for all real numbers except the points \hfill{\brak{2005S}}
    \begin{enumerate}
     \item $\{0,1,-1\}$
     \item $\pm 1$
     \item $1$
     \item $-1$\\
    \end{enumerate}

\item %28

	If $f\brak{x}$ is a continuous and differentiable function and $f\brak{1/n}=0\ \forall\ n\geq1$ and$ n\in I$, then   \hfill{\brak{2005S}}
    \begin{enumerate}
	    \item $f\brak{x}=0,x\in(0,1]$
	    \item $f\brak{0}=0,f'\brak{0}=0$
	    \item $f\brak{0}=0=f'\brak{0}, x\in(0,1]$
	    \item $f\brak{0}=0$ and $f'\brak{0}$ need not be zero\\
    \end{enumerate}


\item %29

	The value of $\lim_{x \to 0}$ $\brak{ \brak{\sin}^{1/x} + \brak{1+x}^{sinx}}$, where $x > 0$ is \hfill{\brak{2006-3M,-1}}
    \begin{enumerate}
     \item $0$
     \item $-1$
     \item $1$
     \item $2$\\
    \end{enumerate}

\newpage
\item %30

	Let $f\brak{x}$ be differentiable on the interval $\brak{0,\infty}$ such that $f\brak{1}=1$, and $\lim_{t \to x} \frac{t^2f\brak{x}-x^2f\brak{t}}{t-x}=1$ for each $x>0$. Then $f\brak{x}$ is \hfill{\brak{2007-3marks}}
    \begin{enumerate}
     \item $\frac{1}{3x}+\frac{2x^2}{3}$\\
     \item $\frac{-1}{3x}+\frac{4x^2}{3}$\\
     \item $\frac{-1}{x}+\frac{2}{x^2}$\\
     \item $\frac{1}{x}$\\
    \end{enumerate}


\item %31

	$\lim_{x \to \frac{\pi}{4}}$$ \frac{\int_2^{\sec^2 x} f(t) dt}{x^2-\frac{\pi^2}{16}}$ equals \hfill{\brak{2007-3marks}}
    \begin{enumerate}
	    \item $\frac{8}{\pi}f\brak{2}$
     \item $\frac{2}{\pi}f\brak{2}$
     \item $\frac{2}{\pi}f\brak{\frac{1}{2}}$
     \item $4f\brak{2}$\\
    \end{enumerate}


\end{enumerate}

\end{document}
