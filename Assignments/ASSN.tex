%iffalse
\let\negmedspace\undefined
\let\negthickspace\undefined
\documentclass[journal,12pt,twocolumn]{IEEEtran}
\usepackage{cite}
\usepackage{amsmath,amssymb,amsfonts,amsthm}
\usepackage{algorithmic}
\usepackage{graphicx}
\usepackage{textcomp}
\usepackage{xcolor}
\usepackage{txfonts}
\usepackage{listings}
\usepackage{enumitem}
\usepackage{mathtools}
\usepackage{gensymb}
\usepackage{comment}
\usepackage[breaklinks=true]{hyperref}
\usepackage{tkz-euclide}
\usepackage{listings}
\usepackage{gvv}                                        
%\def\inputGnumericTable{}                                
\usepackage[latin1]{inputenc}                                
\usepackage{color}                                            
\usepackage{array}                                            
\usepackage{longtable}                                      
\usepackage{calc}                                            
\usepackage{multirow}                                        
\usepackage{hhline}                                          
\usepackage{ifthen}                                          
\usepackage{lscape}
\usepackage{tabularx}
\usepackage{array}
\usepackage{float}


\newtheorem{theorem}{Theorem}[section]
\newtheorem{problem}{Problem}
\newtheorem{proposition}{Proposition}[section]
\newtheorem{lemma}{Lemma}[section]
\newtheorem{corollary}[theorem]{Corollary}
\newtheorem{example}{Example}[section]
\newtheorem{definition}[problem]{Definition}
\newcommand{\BEQA}{\begin{eqnarray}}
\newcommand{\EEQA}{\end{eqnarray}}
\newcommand{\define}{\stackrel{\triangle}{=}}
\theoremstyle{remark}
\newtheorem{rem}{Remark}

% Marks the beginning of the document
\begin{document}
\bibliographystyle{IEEEtran}
\vspace{3cm}

\title{11. Limits, Continuity and Differentiability}
\author{EE24BTECH11047 - Niketh Prakash Achanta}
\maketitle
\newpage
\bigskip
\section{C: MCQs with one correct answer}

\renewcommand{\thefigure}{\theenumi}
\renewcommand{\thetable}{\theenumi}
\begin{enumerate}
\item %17

	$\lim_{x\to0}$  $\frac{\sin\brak{\pi \cos^2 x }}{x^2}$ \: equals \hfill{(2001S)}
    \begin{enumerate}
     \item $-\pi$
     \item $\pi$
     \item $\pi/2$
     \item $1$\\
     \end{enumerate}

\item %18
	The left-hand derivative of f(x)=[x]$\sin\brak{\pi x}$ at\ x=k, k an integer, is \hfill{(2001S)}
    \begin{enumerate}
	    \item $\brak{-1}^k \brak{-1} \pi$
	    \item $\brak{-1}^{k-1}\brak{k-1} \pi$
            \item $-1^k k\pi$
     	    \item $-1^{k-1} k\pi$\\
    \end{enumerate}

\item %19

    Let $f:R\rightarrow R$ be a function defined by f(x)=max$\{x,x^3\}$.\ The set of all points where f(x) is NOT differentiable is \hfill{(2001S)}
    \begin{enumerate}
     \item $\{-1,1\}$
     \item $\{-1,0\}$
     \item $\{0,1\}$
     \item $\{-1,0,1\}$\\
    \end{enumerate}


\item %20

    Which of the following functions is differentiable at x=0 \hfill{(2001S)}
    \begin{enumerate}
	    \item $\cos\brak{\abs{x}}+\abs{x}$
	    \item $\cos\brak{\abs{x}}-\abs{x}$
	    \item $\sin\brak{\abs{x}}+\abs{x}$
	    \item $\sin\brak{\abs{x}}-\abs{x}$\\
    \end{enumerate}


\item %21

    The domain of the derivative of the function $f(x)= 
\begin{array}{ll}
tan^{-1}x
	& \text{if } \abs{x} \leq 0 \\
	\frac{1}{2}(\abs{x}-1) & \text{if } x > 1
\end{array}$ is \hfill{(2002S)}
    \begin{enumerate}
     \item $R-\{0\}$
     \item $R-\{1\}$
     \item $R-\{-1\}$
     \item $R-\{-1,1\}$\\
    \end{enumerate}


\item %22

	The integer n for which $\lim_{x \to 0}\frac{\brak{cosx-1}\brak{cosx-e^{x}}}{x^{n}}$ is a finite non-zero number is \hfill{(2002S)}
    \begin{enumerate}
     \item $1$
     \item $2$
     \item $3$
     \item $4$\\
    \end{enumerate}

\item %23

	Let $f:R \rightarrow R$ be such that $f(1)=3$ and $f'(1)=6$. Then $\lim_{x \to 0}$ $ \brak{ \frac{f(1+x)}{f(1)}}^{1/x}$\ \\ equals \hfill{(2002S)}
    \begin{enumerate}
     \item $1$
     \item $e^{1/2}$
     \item $e^2$
     \item $e^3$\\
    \end{enumerate}


\item %24

	If $\lim_{x \to 0} \frac{\brak{\brak{a-n}nx-tanx}sinnx}{x^2} =0,$ where n is nonzero real number, then a is equal\\ to\hfill{(2003S)}
    \begin{enumerate}
     \item $0$
     \item $\frac{n+1}{n}$
     \item $n$
     \item $n+\frac{1}{n}$\\
    \end{enumerate}


\item %25

    $\lim_{h \to 0} \frac{f(2h+2+h^2)-f(2)}{f(h-h^2+1)-f(1)}$ given that f'(2)=6 and f'(1)=4 \hfill{(2003S)}
    \begin{enumerate}
     \item Does not exist
     \item is equal to -3/2
     \item is equal to 3/2
     \item is equal to 3\\
    \end{enumerate}


\item %26

If f(x) is differentiable and strictly increasing function, then the value of $\lim_{x \to 0} \frac{f(x^2)-f(x)}{f(x)-f(0)}$\\ is \hfill{(2004S)}
    \begin{enumerate}
	    \item $\brak{-1}^k \brak{k-1} \pi$
	    \item $\brak{-1}^{k-1} \brak{k-1} \pi$
	    \item $\brak{-1}^k k\pi$
            \item $\brak{-1}^{k-1} k\pi$\\
    \end{enumerate}


\item %27

	The function given by $y=\abs{\abs{x}-1}$ is differentiable for all real numbers except the points \hfill{(2005S)}
    \begin{enumerate}
     \item $\{0,1,-1\}$
     \item $\pm 1$
     \item $1$
     \item $-1$\\
    \end{enumerate}

\item %28

If f(x) is a continuous and differentiable function and $f(1/n)=0\ \forall\ n\geq1\ and\ n\in I$, then   \hfill{(2005S)}
    \begin{enumerate}
     \item $f(x)=0,x\in(0,1]$
     \item $f(0)=0,f'(0)=0$
     \item $f(0)=0=f'(0), x\in(0,1]$
     \item $f(0)=0$ and $f'(0)$ need not be zero\\
    \end{enumerate}


\item %29

	The value of $\lim_{x \to 0}$ $\brak{ \brak{sin}^{1/x} + \brak{1+x}^{sinx}}$, where $x > 0$ is \hfill{(2006-3M,-1)}
    \begin{enumerate}
     \item $0$
     \item $-1$
     \item $1$
     \item $2$\\
    \end{enumerate}


\item %30

	Let f(x) be differentiable on the interval $\brak{0,\infty}$ such that $f(1)=1$, and $\lim_{t \to x} \frac{t^2f(x)-x^2f(t)}{t-x}=1$ for each $x>0$. Then f(x) is \hfill{(2007-3marks)}
    \begin{enumerate}
     \item $\frac{1}{3x}+\frac{2x^2}{3}$\\
     \item $\frac{-1}{3x}+\frac{4x^2}{3}$\\
     \item $\frac{-1}{x}+\frac{2}{x^2}$\\
     \item $\frac{1}{x}$\\
    \end{enumerate}


\item %31

$\lim_{x \to \frac{\pi}{4}}$$ \displaystyle \frac{\int_2^{sec^2x} f(t) dt}{x^2-\frac{\pi^2}{16}}$ equals \hfill{(2007-3marks)}
    \begin{enumerate}
     \item $\frac{8}{\pi}f(2)$
     \item $\frac{2}{\pi}f(2)$
     \item $\frac{2}{\pi}f(\frac{1}{2})$
     \item $4f(2)$\\
    \end{enumerate}


\end{enumerate}

\end{document}
