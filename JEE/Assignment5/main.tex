\let\negmedspace\undefined
\let\negthickspace\undefined
\documentclass[journal]{IEEEtran}
\usepackage[a5paper, margin=10mm, onecolumn]{geometry}
%\usepackage{lmodern} % Ensure lmodern is loaded for pdflatex
\usepackage{tfrupee} % Include tfrupee package

\setlength{\headheight}{1cm} % Set the height of the header box
\setlength{\headsep}{0mm}     % Set the distance between the header box and the top of the text

\usepackage{gvv-book}
\usepackage{gvv}
\usepackage{cite}
\usepackage{amsmath,amssymb,amsfonts,amsthm}
\usepackage{algorithmic}
\usepackage{graphicx}
\usepackage{textcomp}
\usepackage{xcolor}
\usepackage{txfonts}
\usepackage{listings}
\usepackage{enumitem}
\usepackage{mathtools}
\usepackage{gensymb}
\usepackage{comment}
\usepackage[breaklinks=true]{hyperref}
\usepackage{tkz-euclide} 
\usepackage{listings}
% \usepackage{gvv}                                        
\def\inputGnumericTable{}                                 
\usepackage[latin1]{inputenc}                                
\usepackage{color}                                            
\usepackage{array}                                            
\usepackage{longtable}                                       
\usepackage{calc}                                             
\usepackage{multirow}                                         
\usepackage{hhline}                                           
\usepackage{ifthen}                                           
\usepackage{lscape}


\renewcommand{\thefigure}{\theenumi}
\renewcommand{\thetable}{\theenumi}
\setlength{\intextsep}{10pt} % Space between text and floats


\numberwithin{equation}{enumi}
\numberwithin{figure}{enumi}
\renewcommand{\thetable}{\theenumi}


% Marks the beginning of the document
\begin{document}
\bibliographystyle{IEEEtran}
\vspace{3cm}

\title{jee-main-maths-09-04-2024-shift-2}
\author{EE24BTECH11047 - Niketh Prakash Achanta}
% \maketitle
% \newpage
% \bigskip
{\let\newpage\relax\maketitle}
\renewcommand{\thefigure}{\theenumi}
\renewcommand{\thetable}{\theenumi}

\begin{enumerate}

%1
    \item Let the range of the function $f\brak{x}=\frac{1}{2+\sin{3x}+\cos{3x}},x\in\mathbf{R}$ be $\sbrak{a, b}$. If $\alpha$ and $\beta$ are respectively the arithmetic mean and the geometric mean of $a$ and $b$, then $\frac{\alpha}{\beta}$ is equal to:
    \begin{enumerate}
        \item $\pi$
        \item $\sqrt{\pi}$
        \item $2$
        \item $\sqrt{2}$
    \end{enumerate}
%2
    \item If an unbiased die is rolled thrice, then the probability of getting a greater number in the $i^{th}$ roll than the number obtained in the $\brak{i-1}^{th}$ roll for $i = 2, 3$, is equal to:
    \begin{enumerate}
        \item $2/54$
        \item $5/54$
        \item $1/54$
        \item $3/54$
    \end{enumerate}
%3
    \item Let the foci of a hyperbola H coincide with the foci of the ellipse E:$\frac{\brak{x-1}^2}{100} + \frac{\brak{y-1}^2}{75} = 1$ and the eccentricity of $H$ be the reciprocal of the eccentricity of the ellipse E. If the length of the transverse axis of $H$ is $\alpha$ and the length of its conjugate axis is $\beta$, then $3\alpha^2 + 2\beta^2$ is equal to
    \begin{enumerate}
        \item 225
        \item 205
        \item 237
        \item 242
    \end{enumerate}
%4 
    \item Let $\int_0^x\sqrt{1-\brak{y^{\prime}\brak{t}}^2}dt, 0\leq x \leq 3,y\geq 0,y\brak{0}=0.$ Then at $x=2,y^n+y+1$ is equal to
    \begin{enumerate}
        \item 2
        \item $\sqrt{2}$
        \item 1/2
        \item 1
    \end{enumerate}
%5
    \item The sum of the coefficients of $x^{2/3}$ and $x^{-2/5}$ in the binomial expansion of $\brak{x^{2/3}+\frac{1}{2}x^{-2/5}}^9$ is
    \begin{enumerate}
        \item $19/4$
        \item $69/16$
        \item $63/16$
        \item $21/4$
    \end{enumerate}
%6
    \item The value of the integral $\int_{-1}^{2} \log{\brak{x + \sqrt{x^{2}+1}}} dx$ is
    \begin{enumerate}
        \item $\sqrt{5}-\sqrt{2}+\log{\brak{\frac{9+4\sqrt{5}}{1+\sqrt{2}}}}$
        \item $\sqrt{2}-\sqrt{5}+\log{\brak{\frac{9+4\sqrt{5}}{1+\sqrt{2}}}}$
        \item $\sqrt{5}-\sqrt{2}+\log{\brak{\frac{7+4\sqrt{5}}{1+\sqrt{2}}}}$
        \item $\sqrt{2}-\sqrt{5}+\log{\brak{\frac{7+4\sqrt{5}}{1+\sqrt{2}}}}$
    \end{enumerate}
%7
    \item $\lim_{x \to \frac{\pi}{2}} \brak{ \frac{\int_{x^3}^{\brak{ \frac{\pi}{2} }^3} \brak{ \sin \brak{ 2t^{1/3} } + \cos \brak{ t^{1/3} } } \, dt }{\brak{x - \frac{\pi}{2}}^2} }$ is equal to
    \begin{enumerate}
        \item $\frac{9\pi^2}{8}$
        \item $\frac{3\pi^2}{2}$
        \item $\frac{11\pi^2}{10}$
        \item $\frac{5\pi^2}{9}$
    \end{enumerate}
%8
    \item Let $a,ar,ar^2,\cdots$ be an infinite G.P. If $\sum_{n=0}^{\infty}ar^n=57$ and $\sum_{n=0}^{\infty}a^3r^{3n}=9747$, then $a+18r$ is equal to 
    \begin{enumerate}
    	\item 27
    	\item 31
    	\item 46
    	\item 38
    \end{enumerate}
%9
    \item If $\log_e y = 3\arcsin{x}$, then $\brak{1-x^2}y^{\prime\prime}-xy^{\prime}$ at $x = 1/2$ is equal to:
    \begin{enumerate}
        \item $3e^{\pi/2}$
        \item $9e^{\pi/6}$
        \item $9e^{\pi/2}$
        \item $3e^{\pi/6}$
    \end{enumerate}
%10
    \item $\lim_{x \to 0} \frac{e-\brak{1+2x}^{1/2x}}{x}$ is equal to
    \begin{enumerate}
        \item 0
        \item $-2/e$
        \item $e$
        \item $e-e^2$
    \end{enumerate}
%11 
    \item Let $\vec{a} = 2\hat{i} + \alpha\hat{j} + \hat{k}, \vec{b} = -\hat{i} + \hat{k}, \vec{c}=\beta\hat{j}-\hat{k},$ where $\alpha$ and $\beta$ are integers and $\alpha\beta=-6$. Let the values of the ordered pair$\brak{\alpha,\beta}$, for which the area of the parallelogram of diagonals $\vec{a}+\vec{b}$ and $\vec{b}+\vec{c}$ is $\frac{\sqrt{21}}{2}$, be $\brak{\alpha_1,\beta_1}$ and $\brak{\alpha_2,\beta_2}$. Then $\alpha_1^2+\beta_1^2-\alpha_2\beta_2$ is equal to
    \begin{enumerate}
        \item $17$
        \item $24$
        \item $19$
        \item $21$
    \end{enumerate}
%12
    \item Between the following two statements:\\
    Statement 1: Let $\vec{a}=\hat{i}+2\hat{j}-3\hat{k}$ and $\vec{b}=2\hat{i}+\hat{j}-\hat{k}$. Then the vector $\vec{r}$ satisfying $\vec{a}\times\vec{r}=\vec{a}\times\vec{b}$ and $\vec{a}\cdot\vec{r}=0$ is of magnitude $\sqrt{10}$.\\
    Statement 2: In a triangle ABC, $\cos{2A}+\cos{2B}+\cos{2C}\geq -\frac{3}{2}.$
    \begin{enumerate}
        \item Both Statement 1 and Statement 2 are correct.
        \item Both Statement 1 and Statement 2 are incorrect.
        \item Statement 1 is correct but Statement 2 is incorrect.
        \item Statement 1 is incorrect but Statement 2 is correct.
    \end{enumerate}
%13
    \item Let z be a complex number such that the real part of $\frac{z-2i}{z+2i}$ is zero. Then, the maximum value of $\abs{z-\brak{6+8i}}$ is equal to
    \begin{enumerate}
        \item 10
        \item $\infty$
        \item 8
        \item 12
    \end{enumerate}
%14
    \item If the variance of the frequency distribution \\
   \begin{tabular}{|c|c|c|c|c|c|c|} 
        \hline
            $x$ & c & 2c & 3c & 4c & 5c & 6c \\ 
        \hline
            $f$ & 2 & 1 & 1 & 1 & 1 & 1 \\ 
        \hline
    \end{tabular}\\
    is 160, then the value of $c\in\mathbf{N}$ is
    \begin{enumerate}
        \item 5
        \item 6
        \item 8
        \item 7
    \end{enumerate}
%15
   \item Let $a, b ;a > b$, be the roots of the equation $x^2 - \sqrt{2}x - \sqrt{3} = 0$. Let $P_n = a^n-b^n$, $n\in\mathbf{N}$. Then $(11\sqrt{3} - 10\sqrt{2})P_{10} + (11\sqrt{2} + 10)P_{11} - 11P_{12}$ is equal to:
    \begin{enumerate}
        \item $10\sqrt{2} P_9$
        \item $10\sqrt{3} P_9$
        \item $11\sqrt{2} P_9$
        \item $11\sqrt{3} P_9$
    \end{enumerate}

\end{enumerate}
\end{document}
