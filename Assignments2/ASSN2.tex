%iffalse
\let\negmedspace\undefined
\let\negthickspace\undefined
\documentclass[journal,12pt,twocolumn]{IEEEtran}
\usepackage{cite}
\usepackage{amsmath,amssymb,amsfonts,amsthm}
\usepackage{algorithmic}
\usepackage{graphicx}
\usepackage{textcomp}
\usepackage{xcolor}
\usepackage{txfonts}
\usepackage{listings}
\usepackage{enumitem}
\usepackage{mathtools}
\usepackage{gensymb}
\usepackage{comment}
\usepackage[breaklinks=true]{hyperref}
\usepackage{tkz-euclide} 
\usepackage{listings}
\usepackage{gvv}                                        
%\def\inputGnumericTable{}                                 
\usepackage[latin1]{inputenc}                                
\usepackage{color}                                            
\usepackage{array}                                            
\usepackage{longtable}                                       
\usepackage{calc}                                             
\usepackage{multirow}                                         
\usepackage{hhline}                                           
\usepackage{ifthen}                                           
\usepackage{lscape}
\usepackage{tabularx}
\usepackage{array}
\usepackage{float}


\newtheorem{theorem}{Theorem}[section]
\newtheorem{problem}{Problem}
\newtheorem{proposition}{Proposition}[section]
\newtheorem{lemma}{Lemma}[section]
\newtheorem{corollary}[theorem]{Corollary}
\newtheorem{example}{Example}[section]
\newtheorem{definition}[problem]{Definition}
\newcommand{\BEQA}{\begin{eqnarray}}
\newcommand{\EEQA}{\end{eqnarray}}
\newcommand{\define}{\stackrel{\triangle}{=}}
\theoremstyle{remark}
\newtheorem{rem}{Remark}

% Marks the beginning of the document
\begin{document}
\bibliographystyle{IEEEtran}
\vspace{3cm}

\title{20. Vector Algebra}
\author{EE24BTECH11047 - Niketh Prakash Achanta}
\maketitle
\newpage
\bigskip
\section{C:MCQs With One Correct Answer}
\renewcommand{\thefigure}{\theenumi}
\renewcommand{\thetable}{\theenumi}
\begin{enumerate}
	\item %41
		Two adjacent sides of a parallelogram ABCD are given by $\vec{AB = 2\hat{i}+10\hat{j}+11\hat{k}}$ and $\vec{AD = \hat{i}+2\hat{j}+2\hat{k}}$ \\
		The side AD is rotated by an acute angle $\alpha$ in the plane of the parallelogram so that AD becomes AD$^{\prime}$. If AD$^{\prime}$ makes a right angle with the side AB, then the cosine of the angle $\alpha$ is given by \hfill{\brak{2010}}\\
\begin{enumerate}
	\item $\frac{8}{9}$
	\item $\frac{\sqrt{17}}{9}$
	\item $\frac{1}{9}$
	\item $\frac{4\sqrt{5}}{9}$\\
\end{enumerate}

        \item %42
		Let $\vec{a=\hat{i}+\hat{j}+\hat{k}}$, $\vec{b=\hat{i}-\hat{j}+\hat{k}}$ and $\vec{c=\hat{i}-\hat{j}-\hat{k}}$ be three vectors. A vector $\vec{v}$ in the plane of $\vec{a}$ and $\vec{b}$ , whose projection on $\vec{c}$ is $\frac{1}{\sqrt{3}}$ , is given by \hfill{\brak{2011}}\\
\begin{enumerate}
	\item $\vec{\hat{i}-3\hat{j}+3\hat{k}}$
	\item $\vec{-3\hat{i}-3\hat{j}-\hat{k}}$
	\item $\vec{3\hat{i}-\hat{j}+3\hat{k}}$
	\item $\vec{\hat{i}+3\hat{j}-3\hat{k}}$\\
\end{enumerate}
       
	 \item %43
		 The point $\vec{P}$ is the intersection of the straight line joining the points $\vec{Q\brak{2,3,5}}$ and $\vec{R\brak{1,-1,4}}$ with the plane $5x-4y-z=1$. If $\vec{S}$ is the foot of the perpendicular drawn from the point $\vec{T\brak{2,1,4}}$ to QR, then the length of the line segment PS is \hfill{\brak{2010}}\\
\begin{enumerate}
	\item $\frac{1}{\sqrt{2}}$           
	\item $\sqrt{2}$                   
        \item $2$           
	\item $2\sqrt{2}$\\ 
\end{enumerate}
\newpage
         \item %44
		 The equation of a plane passing through the line of intersection of the planes $x+2y+3z=2$ and $x-y+z=3$ and at a distance $\frac{2}{\sqrt{3}}$ from the point $\vec{\brak{3,1,-1}}$ is \hfill{\brak{2012}}\\
\begin{enumerate}
        \item $5x-11y+z=17$           
	\item $\sqrt{2}x+y=3\sqrt{2}-1$                   
	\item $x+y+z=\sqrt{3}$           
	\item $x-\sqrt{2}y=1-\sqrt{2}$\\ 
\end{enumerate}

         \item %45 
		 If $\vec{a}$ and $\vec{b}$ are vectors such that $\abs{\vec{a}+\vec{b}}$=$\sqrt{29}$ and $\vec{a\times\brak{2\hat{i}+3\hat{j}+4\hat{k}}}$ = $\vec{\brak{2\hat{i}+3\hat{j}+4\hat{k}}}\times\vec{b}$, then a possible value of $\brak{\vec{a}+\vec{b}}\cdot\vec{\brak{-7\hat{i}+2\hat{j}+3\hat{k}}}$ is \hfill{\brak{2012}}\\
\begin{enumerate}
        \item $0$                             
        \item $3$                           
        \item $4$            
        \item $8$\\          
\end{enumerate}

         \item %46 
		 Let $\vec{P}$ be the image of the point $\vec{\brak{3,1,7}}$ with respect to the plane $x-y+z=3$. Then the equation of the plane passing through $\vec{P}$ and containing the straight line $\frac{x}{1}=\frac{y}{z}=\frac{z}{1}$ \hfill{\brak{JEE Adv. 2016}}\\
\begin{enumerate}
        \item $x+y-3z=0$                             
        \item $3x+z=0$                           
        \item $x-4y+7z=0$            
        \item $2x-y=0$\\          
\end{enumerate}

         \item %47 
		 The equation of the plane passing through the point $\vec{\brak{1,1,1}}$ and perpendicular to the planes $2x+y-2z=5$ and $3x-6y-2z=7$, is \hfill{\brak{JEE Adv. 2017}}\\
\begin{enumerate}
        \item $14x+2y-15z=1$                             
        \item $14x-2y+15z=27$                           
        \item $14x+2y+15z=31$            
        \item $-14x+2y+15z=3$\\          
\end{enumerate}

         \item %48 
		 Let $\vec{O}$ be the origin and let PQR be an arbitrary triangle. The point $\vec{S}$ is such that $\vec{OP}\cdot\vec{OQ}$+$\vec{OR}\cdot\vec{OS}$=$\vec{OR}\cdot\vec{OP}$+$\vec{OQ}\cdot\vec{OS}$=$\vec{OQ}\cdot\vec{OR}$+$\vec{OP}\cdot\vec{OS}$\\
Then the triangle PQR has $\vec{S}$ as its \hfill{\brak{JEE Adv. 2017}}\\
\begin{enumerate}
        \item Centroid                             
        \item Circumcentre                           
        \item Incentre            
        \item Orthocenter\\          
\end{enumerate}
\end{enumerate}
 \section{D: MCQs with One or More than One Correct}
\begin{enumerate}
	\item %1
		Let $\vec{a=a_1\hat{i}+a_2\hat{j}+a_3\hat{k}}$, $\vec{b=b_1\hat{i}+b_2\hat{j}+b_3\hat{k}}$ and $\vec{c=c_1\hat{i}+c_2\hat{j}+c_3\hat{k}}$ be three non-zero vectors such that $\vec{c}$ is a unit vector perpendicular to both the vectors $\vec{a}$ and $\vec{b}$. If the angle between $\vec{a}$ and $\vec{b}$ is $\frac{\pi}{6}$, then \\
$
	\mydet{
a_1 & a_2 & a_3 \\
b_1 & b_2 & b_3 \\
c_1 & c_2 & c_3
}^2
$
		is equal to \hfill{\brak{1986-2 Marks}}\\
		\begin{enumerate}
			\item $0$
			\item $1$
			\item $\frac{1}{4}\brak{{a_1}^2+{a_2}^2+{a_3}^2}\brak{{b_1}^2+{b_2}^2+{b_3}^2}$
			\item $\frac{3}{4}\brak{{a_1}^2+{a_2}^2+{a_3}^2}\brak{{b_1}^2+{b_2}^2+{b_3}^2}\brak{{c_1}^2+{c_2}^2+{c_3}^2}$
		\end{enumerate}
\item %2
	The number of vectors of unit length perpendicular to vectors $\vec{a=\cbrak{1,1,0}}$ and $\vec{b=\cbrak{0,1,1}}$ is \hfill{\brak{1987-2Marks}}\\
		\begin{enumerate}
			\item one
		        \item two
			\item three
			\item infinite
			\item None of these
		\end{enumerate}
\item %3
	Let $\vec{a=2\hat{i}-\hat{j}+\hat{k}}$, $\vec{b=\hat{i}+2\hat{j}-\hat{k}}$ and $\vec{c=\hat{i}+2\hat{j}-2\hat{k}}$ be three vectors. A vector in the plane of $\vec{b}$ and $\vec{c}$, whose projection on $\vec{a}$ is of magnitude $\sqrt{2/3}$, is: \hfill{\brak{1993-2Marks}}\\
		\begin{enumerate}
			\item $\vec{2\hat{i}+3\hat{j}-3\hat{k}}$
			\item $\vec{2\hat{i}+3\hat{j}+3\hat{k}}$
			\item $\vec{-2\hat{i}-\hat{j}+5\hat{k}}$
			\item $\vec{2\hat{i}+\hat{j}+5\hat{k}}$
		\end{enumerate}
\item %4
	The vector $\vec{\frac{1}{3}\brak{2\hat{i}-2\hat{j}+\hat{k}}}$ is \hfill{\brak{1994}}\\
		\begin{enumerate}
			\item a unit vector
			\item makes an angle with the vector
			\item parallel to the vector $\vec{\cbrak{-\hat{i}+\hat{j}-\frac{1}{2}\hat{k}}}$
			\item perpendicular to the vector $\vec{3\hat{i}+2\hat{j}-2\hat{k}}$
                \end{enumerate}
\item %5
	If $\vec{a=\hat{i}+\hat{j}+\hat{k}}$,$\vec{b=4\hat{i}+3\hat{j}+4\hat{k}}$ and $\vec{c=\hat{i}+\alpha\hat{j}+\beta\hat{k}}$ are linearly dependent vectors and $\abs{c}=\sqrt{3}$, then \hfill{\brak{1998-2Marks}}\\
		\begin{enumerate}
			\item $\alpha=1,\beta=-1$
			\item $\alpha=1,\beta=\pm1$
			\item $\alpha=-1,\beta=\pm1$
			\item $\alpha=\pm1,\beta=1$
		\end{enumerate}
\end{enumerate}
\end{document}
