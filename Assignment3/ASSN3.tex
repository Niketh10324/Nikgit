%iffalse
\let\negmedspace\undefined
\let\negthickspace\undefined
\documentclass[journal,12pt,twocolumn]{IEEEtran}
\usepackage{cite}
\usepackage{amsmath,amssymb,amsfonts,amsthm}
\usepackage{algorithmic}
\usepackage{graphicx}
\usepackage{textcomp}
\usepackage{xcolor}
\usepackage{txfonts}
\usepackage{listings}
\usepackage{enumitem}
\usepackage{mathtools}
\usepackage{gensymb}
\usepackage{comment}
\usepackage[breaklinks=true]{hyperref}
\usepackage{tkz-euclide} 
\usepackage{listings}
\usepackage{gvv}                                        
%\def\inputGnumericTable{}                                 
\usepackage[latin1]{inputenc}                                
\usepackage{color}                                            
\usepackage{array}                                            
\usepackage{longtable}                                       
\usepackage{calc}                                             
\usepackage{multirow}                                         
\usepackage{hhline}                                           
\usepackage{ifthen}                                           
\usepackage{lscape}
\usepackage{tabularx}
\usepackage{array}
\usepackage{float}


\newtheorem{theorem}{Theorem}[section]
\newtheorem{problem}{Problem}
\newtheorem{proposition}{Proposition}[section]
\newtheorem{lemma}{Lemma}[section]
\newtheorem{corollary}[theorem]{Corollary}
\newtheorem{example}{Example}[section]
\newtheorem{definition}[problem]{Definition}
\newcommand{\BEQA}{\begin{eqnarray}}
\newcommand{\EEQA}{\end{eqnarray}}
\newcommand{\define}{\stackrel{\triangle}{=}}
\theoremstyle{remark}
\newtheorem{rem}{Remark}

% Marks the beginning of the document
\begin{document}
\bibliographystyle{IEEEtran}
\vspace{3cm}

\title{20. Vector Algebra}
\author{EE24BTECH11047 - Niketh Prakash Achanta}
\maketitle
\newpage
\bigskip

\renewcommand{\thefigure}{\theenumi}
\renewcommand{\thetable}{\theenumi}
\section{D: MCQs with One or More than One Correct}
\begin{enumerate}
\item %23
	Three lines $L_1:\vec{r=\lambda\hat{i}}$, $\lambda\in$ R\\
		$L_2:\vec{r=\hat{k}+\mu\hat{j}}$, $\mu\in R$ and\\
		$L_3:\vec{r=\hat{i}+\hat{j}+\nu\hat{k}}$, $\nu\in R$\\
		are given. For which point\brak{s} $\vec{Q}$ on $L_2$ can we find a point $\vec{P}$ on $L_1$ and a point $\vec{R}$ on $L_3$ so that $\vec{P,Q \text{and} R}$ are collinear? \hfill{\brak{JEE Adv. 2019}}\\
  \begin{enumerate}
	  \item $\vec{\hat{k}-\frac{1}{2}\hat{j}}$
	  \item $\vec{\hat{k}}$
	  \item $\vec{\hat{k}+\hat{j}}$
	  \item $\vec{\hat{k}+\frac{1}{2}\hat{j}}$
   \end{enumerate}
\end{enumerate}	
\section{E: Subjective Problems}
\begin{enumerate}
	\item From a point $\vec{O}$ inside the triangle ABC, perpendiculars OD,OE,OF are drawn to the sides BC,CA,AB respectively. Prove that the perpendiculars from $\vec{A,B,C}$ to the sides EF,FD,DE are concurrent. \hfill{\brak{1978}}\\

	\item $A_1,A_2,......A_n$ are the vertices of a regular plane polygon with n sides and $\vec{O}$ is its centre. Show that
	$\sum_{i=1}^{n-1}\brak{\vec{OA_i}\times\vec{OA_{i+1}}}= \brak{1-n}\brak{\vec{OA_2}\times\vec{OA_1}}$
		\hfill{\brak{1982-2Marks}}\\

	\item Find all values of $\lambda$ such that $x,y,z\neq\brak{0,0,0}$ and $\brak{\vec{\hat{i}+\hat{j}+3\hat{k}}}x+\brak{\vec{3\hat{i}-3\hat{j}+\hat{k}}}y+\brak{\vec{-4\hat{i}+5\hat{j}}}z=\lambda\brak{\vec{x\hat{i}+y\hat{j}+z\hat{k}}}$ where $\hat{i},\hat{j},\hat{k}$ are unit vectors along the coordinate axes. \hfill{\brak{1982-3Marks}}\\

	\item A vector $\vec{A}$ has components $A_1,A_2,A_3$ in a right-handed rectangular Cartesian coordinate system oxyz. The coordinate system is rotated about the x-axis through an angle $\frac{\pi}{2}$. Find the components of A in the new coordinate system, in terms of $A_1,A_2,A_3$. \hfill{\brak{1983-2Marks}}\\

	\item The position vectors of the points $\vec{A,B,C \text{and} D}$ are $\vec{\brak{3\hat{i}-2\hat{j}-\hat{k}}}$,$\vec{\brak{2\hat{i}+3\hat{j}-4\hat{k}}}$,$\vec{\brak{-\hat{i}+\hat{j}+2\hat{k}}}$ and $\vec{\brak{4\hat{i}+5\hat{j}+\lambda\hat{k}}}$, respectively. If the points $\vec{A,B,C \brak{and} D}$ lie on a plane, find the value of $\lambda$. \hfill{\brak{1986-2.5Marks}}\\

	\item If $\vec{A,B,C,D}$ are any four points in space, prove that- \hfill{\brak{1987-2Marks}}\\
		$\abs{\vec{AB}\times\vec{CD}+\vec{BC}\times\vec{AD}+\vec{CA}\times\vec{BD}}=4$\brak{area of triangle ABC}\\

	\item Let OACB be a parallelogram with $\vec{O}$ at the origin and OC a diagonal. Let $\vec{D}$ be the midpoint of OA. Using vector methods prove that BD and CO intersect in the same ratio. Determine this ratio. \hfill{\brak{1988-3Marks}}\\ 

\item If vectors $\vec{a},\vec{b},\vec{c}$ are coplanar, show that
	$
		\mydet {
\vec{a} & \vec{b} & \vec{c} \\
\vec{a}\cdot\vec{a} & \vec{a}\cdot\vec{b} & \vec{a}\cdot\vec{c} \\
\vec{b}\cdot\vec{a} & \vec{b}\cdot\vec{b} & \vec{b}\cdot\vec{c}
 }=\vec{0}
		$ \hfill{\brak{1989-2Marks}}\\

	\item In a triangle OAB, $\vec{E}$ is the midpoint of BO and $\vec{D}$ is a point on AB such that AD:DB=2:1. If OD and AE intersect at $\vec{P}$, determine the ratio OP:PD  using vector methods. \hfill{\brak{1989-4 Marks}}\\

	\item Let $\vec{A=2\hat{i}+\hat{k}},\vec{B=\hat{i}+\hat{j}+\hat{k}}$, and $\vec{C=4\hat{i}-3\hat{j}+7\hat{k}}$. Determine a vector $\vec{R}$ satisfying $\vec{R\times B}=\vec{C\times B}$ and $\vec{R}\cdot\vec{A}=0$ \hfill{\brak{1990-3Marks}}\\

	\item Determine the value of 'c' so that for all real $x$, the vector $\vec{cx\hat{i}-6\hat{j}-3\hat{k}}$ and $\vec{x\hat{i}+2\hat{j}+2cx\hat{k}}$ make an obtuse angle with each other. \hfill{\brak{1991-4Marks}}\\

	\item In a triangle ABC,$\vec{D}$ and $\vec{E}$ are points on BC and AC respectively, such that $BD=2DC$ and $AE=3EC$. Let $\vec{P}$ be the point of intersection of AD and BE. Find BP/PE using vector methods. \hfill{\brak{1993-5Marks}}\\

	\item If the vectors $\vec{b,c,d}$ are not coplanar, then prove that the vector $\vec{\brak{a\times b}\times\brak{c\times d}}+\vec{\brak{a\times c}\times\brak{d\times b}}+\vec{\brak{a\times d}\times\brak{b\times c}}$ is parallel to $\vec{a}$. \hfill{\brak{1994-4Marks}}\\

	\item The position vectors of the vertices $\vec{A,B \text{and} C}$ of a tetrahedron ABCD are $\vec{\hat{i}+\hat{j}+\hat{k},\hat{i}}$ and $\vec{3\hat{i}}$ respectively. The altitude from vertex $\vec{D}$ to the opposite face ABC meets the median line through $\vec{A}$ of the triangle ABC at a point $\vec{E}$. If the length of the side AD is $4$ and the volume of the tetrahedron is$\frac{2\sqrt{2}}{3}$, find the position vector of the point $\vec{E}$ for all its possible positions. \hfill{\brak{1996-5Marks}}


\end{enumerate}
\end{document}
